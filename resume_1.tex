% resume.tex
\documentclass[10pt,letterpaper]{article}
\usepackage[letterpaper,margin=0.8in]{geometry}
\usepackage{mdwlist}
\usepackage[T1]{fontenc}
\usepackage{textcomp}
\pagestyle{empty}
\setlength{\tabcolsep}{0em}

\newcommand{\CPP}
{C\nolinebreak[4]\hspace{-.05em}\raisebox{.22ex}{\footnotesize\bf ++}}

\newcommand{\GPP}
{G\nolinebreak[4]\hspace{-.05em}\raisebox{.22ex}{\footnotesize\bf ++}}

\newcommand{\heading}[2]
{\begin{tabular*}{\linewidth}{l@{\extracolsep{\fill}}r}
#1 &
#2 \\
\end{tabular*}}

\author{Eric Dinger}
\begin{document}

%This sets the vertical spacing in the tabular array farther apart so the g isn't squished into the top of the 7
 \renewcommand*\arraystretch{1.5}

\noindent{{\begin{tabular*}{\linewidth}{l@{\extracolsep{\fill}}r}
	\LARGE{\textbf{Eric Dinger}} & \LARGE{{5024 SE 33rd Pl}} \\
	\LARGE{egdinger@gmail.com} & \LARGE{ Portland, OR} \\
	\LARGE{479-244-6432} & \LARGE{97202}
\end{tabular*}}}

\vspace{.2em}
\hrule
\vspace{-.2em}

\subsection*{Education}
	\begin{itemize}
	\item
		\heading
			{Portland State University}
			{Portland, OR}

		\vspace{-1em}
		\heading 
			{ B.S. Computer Science, GPA: 3.45}
			{2012}
	\end{itemize}

\hrule

\subsection*{Open Source Software Experience}
	\begin{itemize}
		\item 
		\heading
			{\textbf{Autonomus Vehicles Team}}
			{Portland State University}
		\begin{itemize*}
			\item Created drivers for sensors: ADXL345 (I2C accelerometer), Maxbotixs sonar sensor, Sharp IR Distance sensor, Autopilot Voltage and Current sensor.
			\item Discovered a bug in the Microbuiler.eu LPC1343 I2C library that caused an malformed stop message to occur during some multi-byte reads. Worked in a team of two to fix the bug.
			\item Created a simple physics simulation of a quadcopter, gyro and accelerometer in python.
			\item Added an option to the make file that launches  and configures GDB to connect to the remote host (Embedded microcontroller) for debugging
			\item Designed and implemented the height measuring subsystem for a quadcopter using a state machine.
		\end{itemize*} 
		\item 
		\heading
			{\textbf{CS Capstone: Linux Kernel Tinification}}
			{Portland State University}
		\begin{itemize*}
			\item Led a capstone (final project) team of 6 students that created several patches to the Linux kernel with the aim of drastically reducing the on disk size for use in embedded environments.
 			\item Patches include: compile time options for core dump removal, tty removal, real time scheduler removal, and changed the command line options for the compression stub to compile time.
			\item Configured KVM based virtual machines used for testing and debugging the modified kernels.
			\item Created testing procedures for the modified kernels.
		\end{itemize*}
	\end{itemize}

\hrule

\subsection*{Work Experience}
	\begin{itemize}
		\item 
		\heading
			{\textbf{Mentor Graphics}}
			{April 2011 - September 2011}
		\emph{Software Engineer Intern}
		\begin{itemize*}
			\item Created a high level programmable interface using TCL for analyzing SVRF rule files inside of YieldServer.
			\item Modify the built-in TCL info command using C++ and wrappers in YieldServer to suppress the return of internal API namespaces. 
		\end{itemize*}
		\item 
		\heading
			{\textbf{FLIR}}
			{April 2010 - September 2010}
			\emph{Software Engineer Intern}
		\begin{itemize*}
			\item Ran Coverity on the code base and reported the findings. Explored how to integrate the use of Coverity into the existing build process.
			\item Worked with manufacturing to design a new tool to set configurations and upload software to the new model Star SAFIRE.
			\item Updated WinSpectrum to use the newest codebase and added the ability to work with NTSC input and output. This required working with Blackmagic Capture Cards and updating the onscreen symbology.
			\item Customer integration of updated WinSpectrum in an unusual networking environment involving serial to Ethernet converters and 9 bit serial protocols.
			\item Created a proof of concept DLL that allowed Labview to communicate with the remote application protocol interface in the new Star SAFIRE.
		\end{itemize*}
	\end{itemize}

\hrule

\subsection*{Personal Projects}
	\begin{itemize}
		\item 
			\heading
				{\textbf{Android RTI Calculator}}
				{2012}
			\vspace{-2em}
			\begin{itemize}
				\item A small app I used to get familiar with the Android environment. RTI is easily compared number relating to suspension performance in offroad trucks, by inputting a few  measurement this app gives you your RTI number.
				\item Awaiting graphic design work before being offered on Google Play.
			\end{itemize}
	\end{itemize}

\hrule

\subsection*{Skills}
	\begin{itemize}
		\item 
		\textbf{Languages}\\
		 C, \CPP, Visual \CPP,  Python, Java, Shell scripting, TCL
		\item
		\textbf{Technologies}\\
		GCC, \GPP{}, GDB, objdump, Android, I2C, Git, grep, Coverity, Windows, Boost \CPP Library's, 
	\end{itemize}

\hrule

\subsection*{Awards \& Honors}
			Mecop Internship\\

\hrule

\subsection*{Clubs \& Activities}
			Portland State Aerospace Society, Viking Motorsports, Autonomous Vehicles Team, IHSTO\\

\end{document}